\cleardoublepage

\hyphenation{mis-iden-tified}

\begin{abstract}
	A search for supersymmetric particles in proton-proton collisions at a \com energy of 13\TeV is presented in this thesis.
	The data were recorded by the CMS experiment during \RunII of the LHC, corresponding to an integrated luminosity of 2.3\fbinv.
	The signal signature comprises pair produced gluon superpartners, gluinos, decaying each to a top-antitop quark pair and a neutralino -- the lightest supersymmetric particle.
	Since the probability, that only one \W boson from the four initial top-quarks decays to a lepton-neutrino pair, is about 40\%, 
	events with a single isolated lepton, electron or muon, accompanied by several high-energy hadronic jets and at least one b-quark tagged jet are selected.
	
	Signal-enriched search regions are defined based on the azimuthal angle between the lepton and four-vector sum of the missing energy and lepton.
	The dominant Standard Model backgrounds tend towards lower angles, while the investigated signals are expected to show little dependence on the angle.
	Further sensitivity to different signal mass scenarios is increased by considering events in categories with different jet and b-tag multiplicities, hadronic and leptonic scales, which are defined as the scalar sums of the jet transverse momenta, and the missing energy and lepton momentum, respectively.
	
	The background is estimated by a data-driven approach using background-enriched control regions.
	Low jet multiplicity data sidebands are used to obtain respective signal-to-control region transfer factors.
	Contributions from multijet events, which arise from misidentified leptons, are subtracted from these control regions using misidentification probabilities measured in data.
	
	No significant deviation from the predicted Standard Model background is observed.
	This result is interpreted in the framework of simplified models corresponding to the aforementioned signal signature, % $\go \to \ttbar \lsp$,
	resulting in exclusion limits for particular gluino-neutralino mass scenarios.
	Gluinos below 1575\GeV are excluded for light neutralinos,
	while for scenarios with low neutralino-gluino mass splitting, neutralinos are excluded up to 850\GeV.
	This extends the limits obtained in \RunI by about 200 and 250\GeV, respectively.
	
	In addition, the upgrade of the \cms \acrlong{ho} readout system with \acrlongpl{sipm} is presented, describing its motivation and the commissioning of the new hardware.
	An initial energy calibration with cosmic muons is performed afterwards.
\end{abstract}
\newpage

\hyphenation{super-sym-metri-schen}
\hyphenation{paar-produk-tion}
\hyphenation{nied-ri-ger}
\hyphenation{aus-schluss-grenzen}
\hyphenation{neu-tra-li-no}

\begin{abstract}[Kurzfassung]
	In dieser Arbeit wird eine Suche nach supersymmetrischen Teilchen in Proton-Proton-Kollisionen bei einer Schwerpunktsenergie von 13 TeV vorgestellt.
	Die Daten wurden vom CMS-Experiment während der zweiten Betriebsperiode des LHC aufgenommen und ent-sprechen einer integrierten Luminosität von 2.3\fbinv. 
	Die untersuchte Signatur ist die Paarproduktion von Gluinos, der Superpartner des Gluons, wobei diese jeweils in ein Top-Antitop-Quark-Paar zerfallen, sowie in ein Neutralino, das leichteste supersymmetrische Teilchen.
	
	Da die Wahrscheinlichkeit, dass nur ein \W-Boson aus dem Zerfall der vier Top-Quarks in ein Lepton-Neutrino-Paar zerfällt, bei rund 40\% liegt, werden Ereignisse mit einem isolierten Lepton (Elektron oder Myon) selektiert, das von mehreren hadronischen Jets begleitet ist.
	Mindestens ein Jet muss als b-Quark-Jet identifiziert sein.
	
	Die Suchbereiche, in denen ein höherer Signalanteil erwartet wird, werden mit Hilfe des Azimuthwinkels zwischen dem Lepton und der Summe aus Leptonimpuls und dem fehlenden Transversalimpuls definiert.
	Der Untergrund durch Prozesse des Standardmodells liegt typischerweise bei kleinen Winkeln, während die untersuchten supersymmetrischen Signale gleichmässig über dem Winkel verteilt sind.
	
	Für verschiedene Szenarien bezüglich der Masse der gesuchten supersymmetrischen Teilchen kann die Empfindlichkeit durch Definition von mehreren Suchkategorien erhöht werden. 
	Dazu werden die Jet- und b-Quark-Jetmultiplizität benutzt, sowie verschiedene Variablen, die ein Maß für die leptonische und hadronische Aktivität im Ereignis sind. Verwendet werden hierzu die skalare Summe der Jet-Transversalimpulse, sowie die Summe aus fehlendem Transversalimpuls und dem Transversalimpuls des Leptons.
	
	Der zu erwartende Standardmodelluntergrund wird aus den Daten abgeschätzt. 
	Hierzu werden Kontrollregionen definiert, in denen Untergrundprozesse überwiegen und Seitenbänder mit wenigen Jets werden verwendet, um die Transferfaktoren zu bestimmen, die das Verhältnis von Signal- zu Kontrollregion beschreiben. 
	Dabei werden Beiträge aus Multijetereignissen, die durch falsch identifizierte Leptonen entstehen, berücksichtigt.
	
	Es wird keine signifikante Abweichung von dem abgeschätzten Standardmodelluntergrund beobachtet. 
	Dieses Ergebnis wird im Rahmen von sogenannten vereinfachten Modellen, entsprechend der oben genannten Signalsignatur, interpretiert,
	was in Ausschlussgrenzen für bestimmte Gluino-Neutralino-Massenszenarien resultiert.
%	Gluinos unter 1575\GeV werden für leichte Neutralinos ausgeschlossen, während für Szenarien mit niedriger Neu-tralino-Gluino-Massendifferenz, 
	Für leichte Neutralinos werden Gluinos unter 1575\GeV ausgeschlossen, während für Szenarien mit niedriger Neutralino-Gluino-Massendifferenz 
	Neutralinos bis 850\GeV ausgeschlossen werden.
	Dies erweitert die in der ersten LHC Betriebsperiode erhaltenen Ausschlussgrenzen um etwa 200 und 250 \GeV.
	
	Darüber hinaus wird das Upgrade des Auslesesystems des äußeren CMS-Hadronkalo-rimeter mit Silizium-Photomultipliern vorgestellt.
	Die entsprechende Motivation sowie die Inbetriebnahme der neuen Hardware werden beschrieben	und die erste Energiekalibrierung mit kosmischen Myonen wird durchgeführt.	
\end{abstract}


%
%\begin{abstract}[Kurzfassung]
%	Eine Suche nach superymmetrischen Teilchen in Proton-Proton Kollisionen bei einer Schwerpunktenergie von 13\TeV wird in dieser Arbeit vorgestellt.
%	Die Daten wurden vom CMS Experiment während des zweiten Laufes des LHC aufgenommen, was einer integrierten Luminositaet von 2.3\fbinv entspricht.
%	Die Signal Signatur bezieht sich auf Paarproduktion von gluon Superpartnern, gluinos, die im nachhinein in ein Top-Antitop Paar und Neutralino -- das leichteste supersymmetrische Teilchen -- zerfallen.
%	
%	Da die Wahrscheinlichkeit, dass nur ein \W Boson von den vier Top-Quarks in ein Lepton-Neutrino Paar zerfaellt bei rund 40\% liegt,
%	werden Ereignisse mit einem isolierten Lepton, Elektron oder Myon, begleitet von mehreren hadronischen Jets und mindestens einem als b-Quark identifizierten Jet selektiert.
%		
%	Signal-angereicherte-Suchbereiche werden auf grund des Azimuthwinkel zwischen dem Lepton und Viervektorsumme der fehlende Energie und des Lepton definiert.
%	Die dominierenden Standard-Modell-Untergründe neigen zu geringen Winkeln,
%	während die untersuchten Signale unabhängig von dem Winkel erwartet werden.
%	Weitere Empfindlichkeit zu unterschiedlichen Signalmassenszenarien wird durch die Kategorisierung von Ereignissen mit exklusiven Jet und b-Tag Multiplizitaeten, hadronischen und leptonischen Skalen, definiert als skalare Summen der Jet Transversalimpulse, und der fehlenden Energie und des Leptonenimpulses, erhöht.
%	
%	Der Untergrund wird einem datengetriebenen Verfahren folgend aus Untergrund-angerei-cherten-Kontrollregionen abgeschätzt.
%	Datenseitenbänder mit wenigen Jets werden verwendet, um entsprechende Signal-zu-Kontroll-Region Transferfaktoren zu erhalten.
%	Beiträge aus Multijet-Ereignissen, die aus falsch identifiziert Leptonen entstehen, werden aus diesen Kontrollregionen anhand in Daten gemessenen 
%%	Misidentifizierungs-Wahrscheinlichkeiten abgezogen.	
%	Misidentifizierungs-Faktoren abgezogen.
%	
%	Keine signifikante Abweichung von dem vorhergesagten Standard-Modell-Hintergrund wird beobachtet.
%	Dieses Ergebnis wird im Rahmen von vereinfachten Modellen, entsprechend der Signalsignatur $\go \to \ttbar \lsp$, interpretiert,
%	was in Ausschlussgrenzen für bestimmte Gluino-Neutralino-Massenszenarien resultiert.
%	Gluinos unter 1575\GeV sind für leichte Neutralinos ausgeschlossen, während für Szenarien mit niedriger Neutra-Gluino-Massendifferenz, Neutralinos bis 850\GeV ausgeschlossen werden.
%	Dies erweitert die im \RunI erhaltenen Ausschlussgrenzen um etwa 200 und 250 \GeV.
%	
%	Darüber hinaus wird das Upgrade des äußeren CMS-Hadronkalorimeter-Auslesesystems mit Silizium-Photomultipliern vorgestellt.
%	Die Motivation und die Inbetriebnahme der neuen Hardware wird beschrieben, und die erste Energiekalibrierung mit kosmischen Myonen wird durchgeführt.
%	
%\end{abstract}
%\newpage

%
%\begin{abstract}[Abstract (shorter)]
%	A search for supersymmetric particles in proton-proton collisions at a \com energy of 13\TeV is presented in this thesis.
%	Data were recorded by the CMS experiment during \RunII of the LHC, corresponding to an integrated luminosity of 2.3\fbinv.
%	The signal signature comprises pair produced gluon superpartners, gluinos, decaying each to a top-antitop quark pair and neutralino -- the lightest supersymmetric particle;
%	one of the top-quarks is considered to decay leptonically via a \W boson.
%	Therefore, events with a single isolated lepton, electron or muon, accompanied by several high-energy hadronic jets and at least one b-quark tagged jet are selected.
%	
%	Signal-enriched search regions are defined based on the azimuthal angle between the lepton and four-vector sum of the missing energy and lepton.
%	The \ttjets and \wjets dominated Standard Model background tends towards lower angles, while signals are expected to show little dependence on the angle.
%	Further sensitivity to different signal mass scenarios is increased by considering events in categories with different jet and b-tag multiplicities, hadronic and leptonic scales, which are defined as the scalar sums of the jet transverse momenta, and the missing energy and lepton momentum, respectively.
%	
%	The background is estimated by a data-driven approach from background-enriched control regions.
%	Low jet multiplicity data sidebands are used to obtain respective signal-to-control region transfer factors.
%	Contributions from multijet events, which arise from misidentified leptons, are subtracted from these control regions using misidentification probabilities measured in data.
%	
%	No significant deviation from the predicted Standard Model background is observed.
%	This result is interpreted in the framework of simplified models corresponding to the signal signature $\go \to \ttbar \lsp$,
%	resulting in exclusion limits for particular gluino-neutralino mass scenarios.
%	Gluinos below 1575\GeV are excluded for light neutralinos,
%	while for scenarios with low neutralino-gluino mass splitting, neutralinos are excluded up to 850\GeV.
%	This extends the limits obtained in \RunI by about 200 and 250\GeV, respectively.
%\end{abstract}
%\newpage
%
%%
%%\begin{abstract}[Abstract (shorter)]
%%A search for supersymmetric particles in proton-proton collisions at a \com energy of 13\TeV is presented in this thesis.
%%Data were recorded by the CMS experiment during \RunII of the LHC, corresponding to an integrated luminosity of 2.3\fbinv.
%%The signal signature comprises pair produced gluon superpartners, gluinos, decaying each to a top-antitop quark pair and neutralino, which is considered as the lightest supersymmetric particle.
%%Events with a single isolated lepton, electron or muon, accompanied by several high-energy hadronic jets and at least one b-quark tagged jet are selected.
%%	
%%	Signal-enriched search regions are defined based on the azimuthal angle between the lepton and a \W boson, reconstructed from the missing energy and lepton four-vectors.
%%	Standard Model events tend towards lower angles, while signals are expected to show little dependence on the angle.
%%	Further sensitivity to different signal mass scenarios is increased by considering events in categories with different jet and b-tag multiplicities, hadronic and leptonic scales, which are defined as the scalar sums of the jet transverse momenta, and the missing energy and lepton momentum, respectively.
%%	
%%	The \ttjets and \wjets dominated Standard Model background is estimated by a data-driven approach from background-enriched control regions.
%%	Low jet multiplicity sidebands in data are used to obtain respective signal-to-control region transfer factors.
%%	The contribution of multijet events, which arises from misidentified leptons, is subtracted from these control regions using misidentification probabilities measured in data.
%%	
%%	No significant deviation from the predicted Standard Model background is observed.
%%	This result is interpreted in the framework of simplified models corresponding to the signal signature $\go \to \ttbar \lsp$,
%%	resulting in exclusion limits for particular gluino-neutralino mass scenarios.
%%	Gluinos below 1575\GeV are excluded for light neutralinos,
%%	while for scenarios with low neutralino-gluino mass splitting, neutralinos are excluded up to 850\GeV.
%%	This extends the limits obtained in \RunI by about 200 and 250\GeV, respectively.
%%\end{abstract}
%%\newpage
%
%\begin{abstract}[Abstract (longer)]
%	A search for supersymmetric particles in proton-proton collisions at a \com energy of 13\TeV is presented in this thesis.
%	Data were recorded by the CMS experiment during \RunII of the LHC, corresponding to an integrated luminosity of 2.3\fbinv.
%	The signal signature comprises pair produced gluon superpartners, gluinos, decaying each to a top-antitop quark pair and neutralino -- the lightest supersymmetric particle;
%	one of the top-quarks is considered to decay leptonically via a \W boson.
%	Therefore, events with a single isolated lepton, electron or muon, accompanied by several high-energy hadronic jets and at least one b-quark tagged jet are selected.
%	
%	The Standard Model background is dominated by \ttjets and \wjets events, where the isolated lepton is originating from \W-boson decays, and therefore aligned with the mother particle.
%	The \W-decay neutrino creates an energy imbalance in the detector, which when summed up with the lepton momentum is related to the \W-boson energy.
%	In case of supersymmetric gluino production, additional momentum imbalance is created by the neutralinos escaping detection, which cancels the angular relation of the lepton and reconstructed \W boson.
%	This feature allows to define signal-enriched regions with suppressed background contamination.
%	Further sensitivity to different signal mass scenarios is increased by considering events in categories with different jet and b-tag multiplicities, hadronic and leptonic scales, which are defined as the scalar sums of the jet transverse momenta, and the missing energy and lepton momentum, respectively.
%	
%	The expected Standard Model background in the search region is estimated by a data-driven approach from
%	background-enriched control regions.
%	Low jet multiplicity sidebands in data are used to obtain respective signal-to-control region transfer factors.
%	The contribution from multijet events, which arises from misidentified leptons, is subtracted from these control regions using misidentification probabilities measured in data.
%	% systematic uncertainties on the prediction are assesed
%	
%	No significant deviation from the predicted Standard Model background is observed.
%	This result is interpreted in the framework of simplified models corresponding to the signal signature $\go \to \ttbar \lsp$,
%	resulting in exclusion limits for particular gluino-neutralino mass scenarios.
%	Gluinos below 1575\GeV are excluded for light neutralinos,
%	while for scenarios with low neutralino-gluino mass splitting, neutralinos are excluded up to 850\GeV.
%	This extends the limits obtained in \RunI by about 200 and 250\GeV, respectively.
%	
%\end{abstract}

\clearpage