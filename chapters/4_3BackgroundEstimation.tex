\chapter{Background estimation}

After the event selection described in a chapter \ref{chap:EventSelection} the background contribution is around 4\% for W-analysis and 0.2\% for Z analysis (which is with this statistics is  neglighable). Main backgrounds for W analysis are coming from:
\begin{itemize}
\item Processes with $\tau$ lepton, misidentified as a electron or muon + missing energy from neutrino
\item Z decays with one missing lepton.
\item QCD processes. In electron channel this is mostly coming from jets faking electrons, while in a muon channel it consists mostly of a real muons produced in decays of heavy-flavor mesons. %The $E_T^{miss}$ distribution is peaking
\end{itemize}
Most of the backgrounds are estimated using MC. They are normalized using highest cross-section order available. The total list of simulated backgrounds and its cross-section is shown in a Table \ref{tab:Backgrounds}. QCD background is estimated using data driven method.

\begin{table}[!bp]
    \caption{Background processes with their associated cross sections and uncertainties. The quoted cross sections are used to normalise estimates of expected number of events}
	\label{tab:Backgrounds}
	\begin{center}
		\begin{tabular}{c | c | c}
		\hline
		\hline
		Process & $\sigma \cdot BR$ [pb] & Order \\
\hline
$W^+ \to l \nu$ & \WPxsec(\WPxsecUncertanty) & NNLO \\ 
$W^- \to l \nu$ & \WMxsec(\WMxsecUncertanty) & NNLO \\ 
\hline
$Z \to ll$ & \Zxsec(\ZxsecUncertanty) & NNLO \\
$Z \to \tau\tau$  & \Zxsec & LO \\
\hline
$t \bar{t}$ & \Ttxsec & LO \\
$WW$ & \WWxsec & LO \\
$ZZ$ & \ZZxsec & LO \\
$WZ$ & \WZxsec & LO \\
$DY \to ee$ & \DYxsec & LO\\
$DY \to \mu\mu$ & \DYxsec & LO \\ 
\hline
\hline
\end{tabular}
\end{center}    
\end{table}


\section{QCD background estimation}\label{sec:QCD}


\begin{figure}[!tbp]
\begin{minipage}[h]{0.49\linewidth}
\center{ \includegraphics[width=1.\linewidth]{QCD/QCDetMissTempl.pdf} \\a)}
\end{minipage}
\hfill
\begin{minipage}[h]{0.49\linewidth}
\center{ \includegraphics[width=1.\linewidth] {QCD/QCDmtWTempl.pdf} \\b)}
\end{minipage}
\caption{Distribution for a) \etmiss b)\mtw template selection  for \wenu events}
\label{ris:TemplateE}

\vfill

\begin{minipage}[h]{0.49\linewidth}
\center{ \includegraphics[width=1.\linewidth]{QCD/QCDWmuetMissTempl.pdf} \\a)}
\end{minipage}
\hfill
\begin{minipage}[h]{0.49\linewidth}
\center{ \includegraphics[width=1.\linewidth] {QCD/QCDWmumtWTempl.pdf} \\b)}
\end{minipage}
\caption{Distribution for a) \etmiss b)\mtw template selection  for \wmunu events}
\label{ris:TemplateMu}
\end{figure}

There is a small probability, that jet can fake W-boson decay with isolated lepton and \etmiss through the energy mismeasurment in the event.  Event selection is suppressing this type of the background, but not fully eliminating it. Due to a large jet production cross-section and complex composition, generation of MC events becomes impractical. This is why data driven technique for QCD background estimation have been used. In our case contribution of the QCD background  in a Z sample is neglighable, so it is estimated just for a \wenu and \wmunu processes. 

Data driven method allows to have model independent predictions with small statistical uncertanty. This method is using \qcd enriched region, where signal events are supressed. This is usually done by reversing identification or isolation criteria. It is assumed, that shape of the qcd background is staying the same in the signal region. Normalization can  be derived in a control region through the template fit. 

This section describes method of QCD background determination, that have been used in 2.76 TeV data. 

\subsection{Template selection}

A study have been performed to determine appropriate template selection. Because of the origins of the QCD backgrounds, missing transverse energy \etmiss should be smaller in a QCD, that in a signal region. Releasing \etmiss cut is allowing to gain a bigger statistics for a QCD template. It is also possible to release \mtw cut. Most of the multijet background event should contribute than in small \mtw region. The template sample can have a contributions from other backgrounds (mostly coming from \wlnu). Best template selection is allowing to have big data statistics and small EWK background contribution at the same time. In order to supress signal additionally reversed ID or Isolation criteria is applied. 

For electron flavor, template is requiring for electron candidate to fail Medium isolation criteria, but to pass loose selection.  Control distribution for a different template selection in electron channel are shown on a Fig. \ref{ris:TemplateE}. Released \etmiss cut is allowing to have a better template statistics. 

In a muon channel template selection build by inverting isolation criteria ( PtCone20 > 0.1). In case of \wmunu the qcd background template can be achieved by releasing \mtw cut (Fig. \ref{ris:TemplateMu}). 

In order to avoid double counting, EWK backgrounds are substracted from a template. The total number of events in the template can be defined as:
\begin{equation}
N_{template} = N^{bkg\, enriched}_{data} - \sum_{j}^{MC} N_{MC_j}^{bkg\, enriched},
\end{equation}
where $N^{bkg enriched}_{data}$ and $N_{MC_j}^{bkg enriched}$ are number of the events in a background enriched sample in data and MC respectivelly. The resulting template statistic is 1348 and 1509 events for \wenu and \wmunu respectively. 


\subsection{Methodology of the template sample normalization}

\begin{figure}[!tbp]
\begin{minipage}[h]{0.49\linewidth}
\center{\includegraphics[width=1.\linewidth]{QCD/QCDetMissFit.pdf} \\ a)}
\end{minipage}
\hfill
\begin{minipage}[h]{0.49\linewidth}
\center{\includegraphics[width=1.\linewidth]{QCD/QCDWmumtWFit.pdf} \\ b)}
\end{minipage}
\caption{Distributions used for multijet background estimation for a) \wenu b)\wmunu}
\label{ris:FitDistributions}
\end{figure}

The normalisation is found through the \chiD fit of the template and backgrounds to the data. The following composite model have been used for estimation:
\begin{equation}
M(x) = \sum_{i=1}^{N-1}f_iF_i(x) + (1- \sum_{i=1}^{N-1} f_i)\cdot F_{qcd}(x),
\end{equation}
where $x$ is a fit variable (\etmiss and \mtw for \wenu and \wmunu respectivelly), $F_i(x)$ and $ F_{qcd}(x)$ are the probability density functions of MC samples and QCD background template respectivelly. Fit parameters $f_i$ are the fractions of MC within fit region. In order to eliminate systematics, coming from cross-section uncertanty, with signal fractions are left freely and and background MC fractions are varied within 5\% uncertanty. 

Normalisation scale for QCD events is calculated from obtained fit parameters as:
\begin{equation}
scale = \frac{(1-\sum f_i) \cdot N^{fit}_{Data}}{N_{template}},
\end{equation}
where $\sum f_i$ is a sum of all fractions in the fit, $N^{fit}_{Data}$ is a number of data events in a fit histogram and $N_{template}$ is a number of event in a template. The fit is performed separatelly for $W^{+}$ and $W^{-}$. Additionally, fit in uncharged $W$ channel is used as a cross-check of the fit. The results of the fitting procedure are shown on a Fig. \ref{ris:Fit} . 
 
\begin{figure}[!tbp]
\begin{minipage}[h]{0.49\linewidth}
\center{\includegraphics[width=1.\linewidth]{QCD/etMissFit.pdf} \\ a)}
\end{minipage}
\hfill
\begin{minipage}[h]{0.49\linewidth}
\center{\includegraphics[width=1.\linewidth]{QCD/MtWFit.pdf} \\ b)}
\end{minipage}
\caption{The multijet background estimation for a) \wenu using reversed ID cut and released \etmiss cut b) \wmunu using released \mtw cut and $b\bar{b}+c\bar{c}$ template}
\label{ris:Fit}
\end{figure}

\subsection{Systematic Uncertainty from the Multi-jet Background Estimation }\label{sec:QCDUnc}

The uncertanty of multi-jet background can esimation can be divided into 3 main components:
\begin{equation}
\delta_{QCD} = \sqrt{ \delta_{fit\, unc}^{2}+\delta_{MC}^{2}+\delta_{fit\, bias}^{2}+\delta_{template}^{2}}, 
\end{equation}
where $\delta_{fit\, unc}$ is the uncertanty for a scale from a \chiD fit. 

The second component $\delta_{MC}$ is coming from a possible mismodelling of MC in a fitted region. It can be estimated by comparison of separate fit results for $W$, $W^{+}$ and $W^{-}$. Number of multijet background events should not depend on a charge of the analysis, so it is expected to have:
\begin{equation}
N_{QCD}^{W}=0.5 \cdot N_{QCD}^{W^{+}} =N_{QCD}^{W^{-}}
\end{equation}
Standard deviation of this 3 numbers is taken as systematic uncertainty. Since in \wmunu channel QCD template is fitted in a region without any additional EWK background this component is 0.

Third uncertanty is due to a potential bias in the template as a result of the template choise and a template statistics itself. For estimation of this uncertainty different template selections have been used. For \wenu channel different reversed isolation criteria have been tried (Fig. \ref{ris:TemplateVar} a)). The overall discrepancies can be considered negligible. For \wmunu channel template variations are estimated using fits with $b\bar{b}+c\bar{c}$ MC samples.  Fig. \ref{ris:TemplateVar} b) compares data template with template obtained using signal selection with released \mtw cut and template selection. Results for a different template fits are presented in a Tab \ref{tab:QCDWmunu}

The third component $\delta_{fit\, bias}$ is coming from an effect from arbirtrary choise of bin size . This error is estimated by repeating fit for a different binnings. This component is assumed negligible in \wmunu case, because of the small number of events. 

Results of QCD background uncertainty estimation for \wenu and \wmunu are shown in a Tab. \ref{tab:QCDWenu} and \ref{tab:QCDWmunu} respectively. The overall number of QCD background events is estimated as \nQCDWplusenu  for $W^{+}\to e^{+}\nu$ and $W^{-}\to e^{-}\nu$ and \nQCDWplusmunu for $W^{+}\to \mu^{+}\nu$ and $W^{-}\to \mu^{-}\nu$. The overall fraction of QCD events is lower, than in 7 TeV data <reference to a 7 TeV paper>, what is agreeing with expectations.

\begin{table}[!tbp]
    \caption{Results of QCD background estimation for \wenu and corresponding error}
	\label{tab:QCDWenu}
	\begin{center}
		\begin{tabular}{c | c | c | c | c }
		\hline
		    Charge & $N_{QCD}$ & $ \delta N_{fit\, unc} $ & $\delta N_{MC}$ & $\delta N_{fit\, bias}$ \\
		    \hline
		    $W^{+}$ & 38.3 & 7.0 & 7.0 & 5.0 \\
		    $W^{-} $ & 21.5 & 0.7 &  -9.4 & 4.0 \\
		    $W$ & 66.1 & 21.2 & 4.2 & 10.  \\
		    \hline
		    \hline
		    Total & 31.0 & 6.1 & 8.6 & 4.7 \\
		    \hline
		\end{tabular}
	\end{center}
\end{table}

\begin{table}[!tbp]
    \caption{Results of QCD background estimation for \wmunu using different templates and it's fit error}
	\label{tab:QCDWmunu}
	\begin{center}
		\begin{tabular}{c | c | c | c  }
		\hline
		    Charge & $N_{QCD}$ & $N_{QCD}$ & $N_{QCD}$ \\
		    & data template & $b\bar{b}+c\bar{c}$ template selection & $b\bar{b}+c\bar{c}$ signal selection \\
		    \hline
		    $W^{+}$ & 2.48 & 0.73 & 1.34 \\
		    $W^{-} $ & 2.48 & 0.73 & 1.35 \\
		    $W$ & 4.97 & 1.47 & 2.70  \\
		    \hline
		    \hline
		    Total per channel & 2.48 & 0.73 & 1.35 \\
		    Fit error & 0.60 & 0.73 & 0.19 \\
		    \hline
		\end{tabular}
	\end{center}
\end{table}



\begin{figure}[!tbp]
\begin{minipage}[h]{0.49\linewidth}
\center{\includegraphics[width=1.\linewidth]{QCD/ElecTemplates.pdf} \\ a)}
\end{minipage}
\hfill
\begin{minipage}[h]{0.49\linewidth}
\center{\includegraphics[width=1.\linewidth]{QCD/QCDBkgWmuTemplates.pdf} \\ b)}
\end{minipage}
\caption{Data and MC comparison for \etmiss calculated by standard \atlas algorithm for a)\wenu b)\wmunu events}
\label{ris:TemplateVar}
\end{figure}

\section{Background-subtracted W and Z candidate events}
Tables \ref{tab:BkgWlnu} and \ref{tab:BkgZll} summarize the number of background events for W and Z selections respectivelly. Uncertanties on a number of EWK+top events are coming from a statistics, cross-section uncertainty (if given) and 3\% of luminosity determination uncertainty. For multijet background uncertainty is coming from a method and described in a subsection \ref{sec:QCDUnc}. For the background-subtracted events the statistical uncertainty is quoted first, followed by the total systematic uncertainty, derived from the EW+top and multijet bacgrkound ones, considering the sources as uncorrelated. 

\begin{table}[!tbp]
    \caption{Numbers of observed candidate events for the $W \to l\nu$ channel, electroweak (EW) plus top, and data- derived QCD background events, and background-subtracted signal events}
	\label{tab:BkgWlnu}
	\begin{center}
		\begin{tabular}{c || c || c | c || c  }
		\hline
		l & Observed & Background & Background & Background-subtracted \\
		 & candidates & (EWK + top) & (Multijet) & data $N_{W}^{sig}$ \\
		 \hline
		 $e^{+}$ & \ntotWplusenu & \nEWttbarbkgWplusenu & \nQCDWplusenu & \ntotsignalWplusenu \\
		 $e^{-}$ & \ntotWminenu & \nEWttbarbkgWminenu & \nQCDWminenu & \ntotsignalWminenu \\
		 $\mu^{+}$ & \ntotWplusmunu & \nEWttbarbkgWplusmunu & \nQCDWplusmunu & \ntotsignalWplusmunu \\
		 $\mu^{-}$ & \ntotWminmunu &\nEWttbarbkgWminmunu & \nQCDWminmunu & \ntotsignalWminmunu \\
		 \hline
		 \end{tabular}
   \end{center}
\end{table}

\begin{table}[!tbp]
    \caption{Numbers of observed candidate events for the $Z \to ll$ channel, electroweak (EW) plus top and background-subtracted signal events}
	\label{tab:BkgZll}
	\begin{center}
		\begin{tabular}{c | c | c | c}
		\hline
		l & Observed & Background & Background-subtracted \\
		 & candidates & (EWK + top)  & data $N_{Z}^{sig}$ \\
		 \hline
		 $e$ & \ntotZee & \nEWttbarbkgZee  & \ntotsignalZee \\
		 $\mu$ & \ntotZmumu &\nEWttbarbkgZmumu  & \ntotsignalZmumu \\
		 \hline
		 \end{tabular}
   \end{center}
\end{table}
