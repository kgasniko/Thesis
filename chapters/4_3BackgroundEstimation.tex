\chapter{Background estimation}
After the event selection described in a chapter \ref{chap:EventSelection} the background contribution is around \% for W and \% for Z analysis (which is with this statistics is  neglighable). Main backgrounds for W analysis are coming from:
\begin{itemize}
\item Processes with $\tau$ lepton, misidentified as a electron or muon + missing energy from neutrino
\item Z decays with one missing lepton.
\item QCD processes. In electron channel this is mostly coming from jets faking electrons, while in a muon channel it consists mostly of a real muons produced in decays of heavy-flavor mesons. %The $E_T^{miss}$ distribution is peaking
\end{itemize}
Most of the backgrounds are estimated using MC. They are normalized using highest cross-section order available. The total list of simulated backgrounds and its cross-section is shown in a Table \ref{tab:Backgrounds}
\begin{table}[h]
    \caption{Background processes with their associated cross sections and uncertainties. The quoted cross sections are used to normalise estimates of expected number of events}
	\label{tab:Backgrounds}
	\begin{center}
		\begin{tabular}{c | c | c}
		\hline
		\hline
		Process & $\sigma \cdot BR (\pm unc.)$ [pb] & Order \\
\hline
$W^+ \to l \nu$ & \WPxsec ($\pm \WPxsecUncertanty$) & NNLO \\ 
$W^- \to l \nu$ & \WMxsec ($\pm \WMxsecUncertanty$) & NNLO \\ 
\hline
$Z \to ll$ & \Zxsec($\pm \ZxsecUncertanty$) & NNLO \\
\hline
$t \bar{t}$ & \Ttxsec & LO \\
$WW$ & \WWxsec & LO \\
$ZZ$ & \ZZxsec & LO \\
$WZ$ & \WZxsec & LO \\
$DY \to ee$ & \DYxsec & LO\\
$DY \to \mu\mu$ & \DYxsec & LO \\ 
\hline
\hline
\end{tabular}
\end{center}    
\end{table}

Due to a large cross-section uncertainty and high amount of events needed to be generated, QCD background is estimated using a data driven method. 
\section{Data-driven method}

\section{Uncertanties estimation}