\chapter{Summary}

This thesis present measurement of  $W\to l\nu$ and $Z\to ll$ cross sections, where $l$ represents electrons or muons. The leptonic decay channels offer a clear experimental signature, that allows to perform a high-precision measurement. Theoretical predictions are available up to next-to-next-to leading order accuracy in QCD and include electroweak (EW) corrections at next-to leading order accuracy. The cross section predictions depend on the parton distribution functions (PDFs) of protons. Therefore, the measurement of W and Z cross sections allows to test models of parton dynamics and offers input for more precise determination of parton distribution functions of proton.

In this analysis the proton-proton sample collected at the \atlas experiment at the center-of-mass energy 2.76 TeV is used. The data corresponds to the integrated luminosity of $4pb^{-1}$. In total about 5500 (and 500) W-boson (and Z-boson) candidates were found per lepton channel.

Due to the very limited size of the sample used, the statistical uncertainty is one of the main uncertainties of this measurement. Several sources of the systematic uncertainties were studied as well. The uncertainty of the luminosity measurement is the dominant one. Studies of the hadronic recoil calibration for the missing transverse energy reconstruction have been performed. The corresponding uncertainties have small, but not negligible, contribution to the systematic uncertainties of the W analyses. The contribution of the main background process has been estimated using Monte Carlo simulation, except for the multijet background for which a data-driven method was used.

The measured $W\to e\nu$ and $Z\to ee$ fiducial cross sections are:
\begin{center}
$\sigma^{fid}_{W}(\wenu) = \valWenu  \pm \statWenu (stat.) \pm \sysWenu (sys.) \pm \lumiWenu (lumi.) \, [pb]$,\\
$\sigma^{fid}_{Z}(Z\to ee) = \valAelecZ  \pm \statAelecZ (stat.) \pm \sysAelecZ (sys.) \pm \lumiAelecZ (lumi.) \, [pb]$,\\
\end{center}
and the cross section for $W\to \mu\nu$ and $Z\to \mu \mu $ are:
\begin{center}
$\sigma^{fid}_{W}(\wmunu) = \valWmunu  \pm \statWmunu (stat.) \pm \sysWmunu (sys.) \pm \lumiWmunu (lumi.) \, [pb]$,\\
$\sigma^{fid}_{Z}(Z\to ee) = \valAmuonZ  \pm \statAmuonZ (stat.) \pm \sysAmuonZ (sys.) \pm \lumiAmuonZ (lumi.) \, [pb]$.\\
\end{center}
The measured values of cross sections agree with each other within the uncertainties. There is also agreement with theoretical predictions at the different orders of QCD accuracy.

\newpage

The measurement of cross-section ratios benefits from the partial cancellation of correlated experimental uncertainties and full cancellation of luminosity uncertainty, which makes them a powerful tool to test SM predictions. The measures cross section ratios in muon to electron channels in fiducial region  are:
\begin{center}
$R_{W}=\frac{\sigma^{\mu}_W}{\sigma^{e}_W} = \frac{BR(\wmunu)}{BR(\wenu)}= \valWunivers \pm \sysWunivers (sys.) \pm \statWunivers (stat.)$,\\
$R_{Z}=\frac{\sigma^{\mu}_Z}{\sigma^{e}_Z} = \frac{BR(Z\to \mu\mu)}{BR(Z \to ee)}= \valZunivers \pm \sysZunivers(sys.) \pm \statZunivers (stat.)$
\end{center}
which is in agreement, within the uncertainty, with SM predictions and the world average. 


The combination of electron and muon channel cross section results allows to further reduce the statistical uncertainties of the measurements: 
\begin{center}
$\sigma^{fid}_{W}(W \to l\nu) =  \valfidWtot \pm \statfidWtot (stat.) \pm \sysfidWtot (sys.) \pm \lumifidWtot (lumi.) \, [pb]$, \\
$\sigma^{fid}_{Z}(Z \to ll) = \valfidZ \pm \statfidZ (stat.) \pm \sysfidZ  (sys.) \pm \lumifidZ (lumi.) \, [pb]$. \\
\end{center}
These results are used to measure W to Z cross section ratios:
\begin{center}
$R_{W/Z} = \valfidWZ \pm \statfidWZ \, (stat.)\, \pm \sysfidWZ\, (sys.)$,\\
\end{center}
which has an uncertainty comparable with the uncertainty on next-to leading order predictions. 

The obtained cross sections have been used to constrain the PDF fits and have shown a slight reduction of uncertainties for $u$, $d$ and $\bar{u}$, $\bar{d}$ distribution functions. The analysis denoted to constraining the PDF distributions would certainly benefit from the bigger data sample collected at similar center-of-mass energy and a combination with measurements of W and Z cross sections at other center-of-mass energies.


