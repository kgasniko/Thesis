\chapter{Additional comparisons with theoretical predictions}

The cross-section results for combined channel in fiducial regions from have been compared to NNLO predictions for different PDF sets in Sec.~\ref{sec:CombCs}. This appendix presents the comparison of cross-section in full and extrapolated to 13 TeV regions with NNLO predictions (Fig.~\ref{fig:AppD1}-Fig.~\ref{fig:AppD2}). The agreement between predictions and results in full region is worse, than for fiducial and extrapolated regions, however it is still within 2$\sigma$ of uncertainty. 

Additionally, the comparison for NLO predictions in fiducial region is presented in Fig.~\ref{fig:AppD3}-~\ref{fig:AppD4}.

\begin{figure}[!h]
\begin{minipage}[h]{0.32\linewidth}
\center{\includegraphics[width=1\textwidth]{Results/NNLOWpfull.pdf} \\ a)}
\end{minipage}
\hfill
\begin{minipage}[h]{0.32\linewidth}
\center{\includegraphics[width=1\textwidth]{Results/NNLOWmfull.pdf} \\ b)}
\end{minipage}
\begin{minipage}[h]{0.32\linewidth}
\center{\includegraphics[width=1.0\textwidth]{Results/NNLOZfull.pdf} \\ c)}
\end{minipage}
\caption{The NNLO predictions for the full cross-section a) $\sigma^{full}_{W^+}$  b) $\sigma^{full}_{W^-}$  c) $\sigma^{full}_Z$ in pb for the six PDFs CT14nnlo, MMHT2014, NNPDF3.0, ATLASepWZ12, abm12, HERApdf2.0 compared to the measured cross-section as given in Tab.~\ref{tab:csComb}. The green (cyan) band corresponds to the experimental uncertainty without (with) the luminosity uncertainty. The theory predictions are given with the corresponding PDF uncertainties shown as error bands.}
\label{fig:AppD1}
\end{figure}

\begin{figure}[!h]
\begin{minipage}[h]{0.32\linewidth}
\center{\includegraphics[width=1\textwidth]{Results/NNLOWp13.pdf} \\ a)}
\end{minipage}
\hfill
\begin{minipage}[h]{0.32\linewidth}
\center{\includegraphics[width=1\textwidth]{Results/NNLOWm13.pdf} \\ b)}
\end{minipage}
\begin{minipage}[h]{0.32\linewidth}
\center{\includegraphics[width=1.0\textwidth]{Results/NNLOZ13.pdf} \\ c)}
\end{minipage}
\caption{The NNLO predictions for the extrapolated to the 13 TeV cross-section a) $\sigma^{13}_{W^+}$  b) $\sigma^{13}_{W^-}$  c) $\sigma^{13}_Z$ in pb for the six PDFs CT14nnlo, MMHT2014, NNPDF3.0, ATLASepWZ12, abm12, HERApdf2.0 compared to the measured cross-section as given in Tab.~\ref{tab:csComb}. The green (cyan) band corresponds to the experimental uncertainty without (with) the luminosity uncertainty. The theory predictions are given with the corresponding PDF uncertainties shown as error bands.}
\label{fig:AppD2}
\end{figure}

\begin{figure}[!h]
\begin{minipage}[h]{0.49\linewidth}
\center{\includegraphics[width=1\textwidth]{Results/NLOWp.pdf} \\ a)}
\end{minipage}
\hfill
\begin{minipage}[h]{0.49\linewidth}
\center{\includegraphics[width=1\textwidth]{Results/NLOWm.pdf} \\ b)}
\end{minipage}
\caption{The NLO predictions for the fiducial cross-section a) $\sigma^{fid}_{W^+}$  b) $\sigma^{fid}_{W^-}$  in pb for the six PDFs CT14nnlo, MMHT2014, NNPDF3.0, ATLASepWZ12, abm12, HERApdf2.0 compared to the measured fiducial cross-section as given in Tab.~\ref{tab:csComb}. The green (cyan) band corresponds to the experimental uncertainty without (with) the luminosity uncertainty. The theory predictions are given with the corresponding PDF uncertainties shown as error bands.}
\label{fig:AppD3}
\end{figure}

\begin{figure}[!h]
\center{\includegraphics[width=0.49\textwidth]{Results/NLOZ.pdf}}
\caption{The NLO predictions for the fiducial cross-section $\sigma^{fid}_Z$ in pb for the six PDFs CT14nnlo, MMHT2014, NNPDF3.0, ATLASepWZ12, abm12, HERApdf2.0 compared to the measured fiducial cross-section as given in Tab.~\ref{tab:csComb}. The green (cyan) band corresponds to the experimental uncertainty without (with) the luminosity uncertainty. The theory predictions are given with the corresponding PDF uncertainties shown as error bands.}
\label{fig:AppD4}
\end{figure}


\chapter{Additional PDF profiling plots}\label{app:PDF}

The results of PDF profiling have been showed in Sec.~\ref{sec:PDFCs}. In this Appendix the effect on valence quarks ratio $d_{v}/u_{v}$ (Fig.~\ref{fig:AppC1} and difference in sea u and d quarks $\bar{d}-\bar{u}$ (Fig.~\ref{fig:AppC2}) is shown. The effect of inclusion of the new data at the scale of the measurement $Q^2 \approx M^2_{W}$ is shown in Fig.~\ref{fig:AppC3}-~\ref{fig:AppC4}. 

\begin{figure}[!h]
\begin{minipage}[h]{0.49\linewidth}
\center{\includegraphics[width=1\textwidth]{Results/Shift/dbar-ubar.pdf} \\ a)}
\end{minipage}
\hfill
\begin{minipage}[h]{0.49\linewidth}
\center{\includegraphics[width=1\textwidth]{Results/Shift/dbar-ubar_ratio.pdf} \\ b)}
\end{minipage}
\caption{The a) absolute and  b) relative distributions for the $\bar{d}-\bar{u}$ quark densities as a function of $x$ at scale $Q^2=$ 1.9 GeV$^2$ with the experimental uncertainties. The red band denotes the reference NLO PDF distributions from CT14 pdf set. The impact of addition of the new W,Z cross-sections at 2.76 TeV on the PDF set is shown by the blue boundaries.}
\label{fig:AppC1}
\end{figure}

\begin{figure}[!h]
\begin{minipage}[h]{0.49\linewidth}
\center{\includegraphics[width=1\textwidth]{Results/Shift/doveru.pdf} \\ a)}
\end{minipage}
\hfill
\begin{minipage}[h]{0.49\linewidth}
\center{\includegraphics[width=1\textwidth]{Results/Shift/dvoveruv_ratio.pdf} \\ b)}
\end{minipage}
\caption{The a) absolute and  b) relative distributions for the $u_v / d_v$ quark densities as a function of $x$ at scale $Q^2=$ 1.9 GeV$^2$ with the experimental uncertainties. The red band denotes the reference NLO PDF distributions from CT14 pdf set. The impact of addition of the new W,Z cross-sections at 2.76 TeV on the PDF set is shown by the blue boundaries}
\label{fig:AppC2}
\end{figure}

\begin{figure}[!h]
\begin{minipage}[h]{0.49\linewidth}
\center{\includegraphics[width=1\textwidth]{Results/Q2Evol/q2_6464_pdf_uv.pdf} \\ a)}
\end{minipage}
\hfill
\begin{minipage}[h]{0.49\linewidth}
\center{\includegraphics[width=1\textwidth]{Results/Q2Evol/q2_6464_pdf_dv.pdf} \\ b)}
\end{minipage}
\caption{The absolute for the a) $u_v$ and b) $d_v$ quark densities as a function of $x$ at scale $Q^2=M_W^2$ with the experimental uncertainties. The red band denotes the reference NLO PDF distributions from CT14 pdf set. The impact of addition of the new W,Z cross-sections at 2.76 TeV on the PDF set is shown by the blue boundaries.}
\label{fig:AppC3}
\end{figure}

\begin{figure}[!h]
\begin{minipage}[h]{0.49\linewidth}
\center{\includegraphics[width=1\textwidth]{Results/Q2Evol/q2_6464_pdf_UBar.pdf} \\ c)}
\end{minipage}
\hfill
\begin{minipage}[h]{0.49\linewidth}
\center{\includegraphics[width=1\textwidth]{Results/Q2Evol/q2_6464_pdf_DBar.pdf} \\ d)}
\end{minipage}
\caption{The absolute for the a) $\bar{u}$ and b) $\bar{d}$ quark densities as a function of $x$ at scale $Q^2=M_W^2$ with the experimental uncertainties. The red band denotes the reference NLO PDF distributions from CT14 pdf set. The impact of addition of the new W,Z cross-sections at 2.76 TeV on the PDF set is shown by the blue boundaries.}
\end{figure}

\begin{figure}[!h]
\begin{minipage}[h]{0.49\linewidth}
\center{\includegraphics[width=1\textwidth]{Results/Q2Evol/q2_6464_pdf_s.pdf} \\ e)}
\end{minipage}
\hfill
\begin{minipage}[h]{0.49\linewidth}
\center{\includegraphics[width=1\textwidth]{Results/Q2Evol/q2_6464_pdf_g.pdf} \\ f)}
\end{minipage}
\caption{The absolute for the a) $s$ quark and b) gluon densities as a function of $x$ at scale $Q^2=M_W^2$ with the experimental uncertainties. The red band denotes the reference NLO PDF distributions from CT14 pdf set. The impact of addition of the new W,Z cross-sections at 2.76 TeV on the PDF set is shown by the blue boundaries.}
\label{fig:AppC4}
\end{figure}
