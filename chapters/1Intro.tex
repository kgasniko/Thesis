\chapter{Thesis organization}
This thesis presents the measurement of $W\to l\nu$ and $Z\to ll$ cross-sections in electron in muon channels in 2.76 TeV data collected by \atlas experiment.

The thesis is organized in a three parts. The theoretical input is described in part 1. The experimental input and a software organization is explained in a part 2. The cross-section measurement performed by the author is described in part 3. The results and its interpretation via parton density functions is presented in a final part.

The work presented was performed within the ATLAS collaboration. All plots in the thesis were produced by the author, unless it is referenced otherwise. 

The theoretical input is presented in part 1 and chapters:
\begin{description}
\item [Chapter 2] \textbf{Theoretical introduction}, contains brief overview of the current status of Standard Model and the proton structure
\item [Chapter 3] \textbf{Methodology}, describes methods of cross-section and their ratios calculation and methods of determination of effect of this cross-section on a PDF distributions.
\end{description}

The experimental setup is desctibed in the part 2 and following chapters:
\begin{description}
\item [Chapter 4] \textbf{The LHC and ATLAS experiment} gives and overview of the LHC accelerator complex and its experiments and the \atlas detector, used to collect data for this analysis
\item [Chapter 5] \textbf{Event reconstruction} сontains the detailed description of event reconstruction. The study of missing transverse energy reconstruction algorithm was presented by author. It was figured out, that for 2.76 TeV data the non-standard procedure is needed. 
\item [Chapter 6] \textbg{Monte-Carlo} provides an information of Monte-Carlo simulation steps and generators, used in this analysis
\item [Chapter 7] \textbg{Frozen showers} gives an description of Frozen Showers method for fast Monte-Carlo simulation. The machine learning method for its optimization, made by author, is presented.
\item [Chapter 8] \textbf{Data and Monte-Carlo samples} describes a data and Monte-Carlo samples, used in the analysis.
\end{description}

The following chapters presents a work, done by author, unless other is specified:
\begin{description}
\item [Chapter 9] \textbf{Event selection} gives a set of selection criteria used to derive $W\to l\nu$ and $Z\to ll$ in collected data
\item [Chapter 10] \textbf{Monte-Carlo corrections} presents the correction, applied on Monte-Carlo in order to gain better data vs Monte-Carlo agreement. The correction factors have been derived by perormance group, except for muon trigger scale factors, determined by the author.
\item [Chapter 11] \textbf{Hadronic recoil calibration} describes a method of missing transverse energy calibration in 2.76 GeV data and methods of the corresponding uncertainty determination.
\item [Chapter 12] \textbf{Background estimation} provides a description of main backgrounds, that can pass the selection criteria and techniques of their contribution estimation
\item [Chapter 13] \textbf{Control distributions} shows the agreement between data and Monte-Carlo simulation for all analyses and different distributions.
\item [Chapter 14] \textbf{Uncertainties of the cross-section measurements} presents main sources of experimental and theoretical uncertainties and gives a methods of their propagation to the final cross-sections and their ratios.
\item [Chapter 15] \textbf{Results of the cross-section measurements} presents the results of cross-section measurement of $W\to l\nu$ and $Z\to ll$ in electron and muon channels separately, results of combined cross-sections and their ratios and effect of this measurment on PDF distributions.
\end{description}